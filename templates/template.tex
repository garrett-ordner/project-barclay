% templates/template.tex
\documentclass[openany]{book}
%Bible Verses
\input{../../bible/KJV.tex}
\usepackage{amsmath}
\usepackage{graphicx}
\usepackage{hyperref}
\usepackage[a4paper]{geometry} % or [letterpaper] if needed
\usepackage{subfiles}
\usepackage{pgffor}
\usepackage{endnotes}

% Paragraph formatting
\setlength{\parindent}{0pt} % Remove paragraph indentation
\setlength{\parskip}{1em}   % Add space between paragraphs

%commands to get bible verse
\newcommand{\getBibleText}[3]{%
	\ifcsname verse#1#2v#3\endcsname%
	\csname verse#1#2v#3\endcsname%
	\else%
	[Verse not found: #1 #2:#3]%
	\fi%
}

\newcommand{\getBibleRange}[4]{%
	\foreach \v in {#3,...,#4} {%
		\getBibleText{#1}{#2}{\v} %
	}%
}

%Use this command to quote verses; you can change the type of quoting used in the book using the commands defined below.
%Single verse
\newcommand{\qVerse}[3]{\enVerse{#1}{#2}{#3}}
%Range
\newcommand{\qRange}[4]{\enRange{#1}{#2}{#3}{#4}}

%Footnote Bible References
\newcommand{\fnVerse}[3]{\footnote{\getBibleText{#1}{#2}{#3}}}
\newcommand{\fnRange}[4]{\footnote{\getBibleRange{#1}{#2}{#3}{#4}}}

%Endnote Bible References
\newcommand{\enVerse}[3]{\endnote{\getBibleText{#1}{#2}{#3}}}
\newcommand{\enRange}[4]{\endnote{\getBibleRange{#1}{#2}{#3}{#4}}}