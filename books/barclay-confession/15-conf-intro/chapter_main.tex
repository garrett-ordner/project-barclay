%Barclay's Catechism Preface
\documentclass[../main.tex]{subfiles}

\begin{document}
	
	\chapter*{Preface}
	\addcontentsline{toc}{chapter}{Preface}
	
	Now that my catechism has given the Scriptural justification for the beliefs of the Quakers, and by using the Scriptures to answer the objections made against us, I provide a Confession of Faith, which I will keep brief, because it would be redundant to reiterate all the Scriptures provided in the Catechism. Rather, a Confession of Faith is meant to be an account of one's own faith, not a debate of them like the Catechism. For clarity, I sometimes need to add words like "and", and "therefore", but not in any way that adds my own commentary to the Confession. In fact, in order to satisfy the most insufferable of my critics, I am even marking these words differently.\footnote{Editor's note: While Barclay originally did mark these words differently, for ease of writing, this work does not mark these words differently, assuming that the good-faith reader will understand that they are not significant.} Similarly, to avoid using nonsensical grammar, I sometimes have to  change first-person to third-person, or the conjugation of a verb, but no one who reads this would ever think I am altering the meaning. For example, when Christ says, "I am the Light of the World," would it be proper for me to write, "I am the Light", or would it not be better to write, "Christ is the Light," where the first-person is changed to the third? Also, sometimes the apostles say, "we", which I write as "the apostles," and when said apostles spoke the saints, they said, "you", which I write as "the saints." I do this because connecting the sentences sometimes requires it, such as the first article, which mentions 1 John 1:5\qVerse{IJohn}{1}{5} concerning God's being Light. In such cases, I know an impartial reader would not argue with these changes, but I also realize the so-called "Christians" of this era, who have no real argument to make against the truth and its followers (the Quakers), will instead nitpick over trivialities at every opportunity. I hope this explanation satisfies those critics.

	\theendnotes
	\setcounter{endnote}{0}

\end{document}