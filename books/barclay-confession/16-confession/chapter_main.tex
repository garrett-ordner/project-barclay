%Barclay's Catechism Chapter 1
\documentclass[../main.tex]{subfiles}

\begin{document}
	
	\chapter*{A Confession of Faith, containing twenty-three Articles}
	\addcontentsline{toc}{chapter}{A Confession of Faith, containing twenty-three Articles}
	
	\section{Concerning God, and the True and Saving Knowledge of Him}

	There is one God [Eph. 4:6, 1 Cor 8:4,6]\qVerse{Ephesians}{4}{6}\textsuperscript{,}\qVerse{ICorinthians}{8}{4}\textsuperscript{,}\qVerse{ICorinthians}{8}{6}, who is a Spirit [John 4:24]\qVerse{John}{4}{24}, and this is the message that the apostles heard from Him and preached to the saints: He is Light, and there is no darkness in Him. [1 John 1:5]\qVerse{IJohn}{1}{5} There are three in heaven: the Father, the Son, and the Holy Spirit, and these three are one [1 John 5:7]\qVerse{IJohn}{5}{7}: The Father is in the Son, and the Son is in the Father. [John 10:38, 14:10-11, 5:26]\qVerse{John}{10}{38}\textsuperscript{,}\qRange{John}{14}{10}{11}\textsuperscript{,}\qVerse{John}{5}{26} No one knows the Father except the Son, and no one knows the Son except the Father, and those to whom the Son reveals Him. [Matt. 11:27, Luke 10:22]\qVerse{Matthew}{11}{27}\textsuperscript{,}\qVerse{Luke}{10}{22} The Spirit knows the deepest things of God [1 Cor. 2:10]\qVerse{ICorinthians}{2}{10}, for the things of God know no man, only the Spirit of God: The saints have not received the spirit of the world, but the Spirit of God, so that they can know the things freely given to them by God. [1 Cor. 2:11-12]\qRange{ICorinthians}{2}{11}{12} The Comforter (the Holy Ghost) who the Father sends in Christ's name, he teaches the saints everything, and brings everything to their remembrance. [John 14:26]

	\section{Concerning the Guide and Rule of Christians}

	Christ prayed to the Father, and He gave the saints another Comforter, which would be with them forever: The Spirit of Truth who the world cannot receive, becuase it does not see Him or know Him, but the saints know him; for He dwells with them and within them. [John 14:16-17] If any man does not have the Spirit of Christ, he does not belong to Christ, for those who are led by the Spirit of God are sons of God [Rom. 8:9,14] For this is the covenant that God made with Israel: he put His laws in their mind, and wrote them on their hearts; and that are all taught of God [Heb. 8:10-11] And the anointing they have received from God abes in them, and they don't need any man to teach them, but the anointing teaches them ass things, and is the truth, not a lie. [1 John 2:27]

	\section{Concerning the Scriptures}

	Whatever was written before, was written for our learning, so that we, through the patience and comfort of the Scriptures, could have hope [Rom. 15:4] that could bring us to salvation through faith in Jesus Christ. All Scripture is inspired by God, and is useful for doctrin, reproof, and instruction in righteousness, so that the man of God can by perfect, capable of all good works [2 Tim. 3:15-17] No prophecy of the Scripture is anyone's private interpretation, because it is not the product of man's will, but the words of holy men of God who spoke as they were moved by the Holy Spirit. [2 Pet. 1:20-21]

	\section{Concerning the Divinity of Christ, and his Being from the Beginning}

	In the beginning, there was the Word, and the Word was with God, and the Word was God; the Word was in the Beginning with God. He made all things, and nothing was made without him. [John 1:1-3] He has always existed [Mic. 5:2] For God created all things by Jesus Christ [Eph. 3:9] who, being in the form of God, was equal with God [Phil 2:6] and His name is called, wonderful counsellor, the mighty God, the everlasting Father, the prince of peace [Isa. 9:6] who is the image of the invisible God, the first of every creature [Col. 1:5], the brightness of His Father's glory, and the image of His Father's substance [Heb. 1:3] whose clothes were dipped in blood, and whose name is called the Word of God [Rev. 19:13] In him dwells the fullness of God [Col. 2:9] and in him are hidden all the treasures of wisdom and knoowledge. [Col. 2:3]

	\section{Concerning his Appearance in the Flesh.}

	The Word was made into flesh [John 1:14] because he did not take on the nature of an angel, but instead became a descendent of Abraham, being as human as anyone else [Heb. 2:16-17], understanding our sufferings, and tempted just as we are, though He did not sin [Heb. 4:15] He died for our sins, according to the Scriptures, and he was buried but rose again the third day, according to the Scriptures. [1 Cor. 15:3-4]

	\section{Concerning the End and Use of that Appearance}

	God sent his own Son as a human, who condemned sin [Rom. 8:3] This is why the Son of God came: to destroy the works of the Devil [1 John 3:8] and to take away our sins [1 John 3:5]. He gave Himself for us, a sweet-smelling offering and a sacrifice to God [Eph. 5:2] He obtained eternal redemption for us [Heb. 9:12] and through the eternal Spirit, offered himself, pure and blameless, to God, in order to cleanse our consciences from dead works, so we could serve the living God. [Heb. 9:14] He was the lamb that was slain from the beginning of the world [Rev. 5:8,12;13:8] All the church fathers drank of him, for they drank of the spiritual rock that was with them, and that rock was Christ [1 Cor. 10:1-4]. Christ also suffered for us to leave us an example that we should follow [1 Pet. 2:21] for in our body we should bear Christ's death, so that Christ's life could also be shown in our body, being delivered unto death for Jesus' sake,  so that our lives may demonstrate Christ's life. [2 Cor. 4:10-11] in order that we can know Him, and the power of his resurrection, and the fellowship of his suffering, having conformed to His death. [Phil. 3:10]

	\section{Concerning the Inward Manifestation of Christ}

	God dwells with the contrite and humble in spirit [Isa. 57:15] because he said he would dwell in them and walk in them [2 Cor. 6:16]. Christ stands at the door and knocks; if any man hears His voice, and opens the door, Christ will come to him, and be with him [Rev. 3:20] Therefore, we need to examine ourselves, and prove our selves, knowing that unless we are reprobates, Christ must be in us [2 Cor. 13:5] For thisis the glory of God's mystery, which God would make known to the Gentiles: Christ within, the hope of glory. [Col 1:27]

	\section{Concerning the New Birth}

	Unless a man is born again, he cannot see the Kingdom of God [John 3:3] Therefore, we need to put away the old man, with his deeds, and put on the new man, renewed in knowledge and in the image of God, and like God is created in righteousness and holiness. [Eph. 4:21-22; Col. 3:10] From know on, we know no worldly man, for though we knew Christ when he was a human, we know him know more [2 Cor. 5:16] Likewise, if any man is in Christ, he is a new creature, and his old ways are gone; all things have become new [2 Cor. 5:17] For he has taken on Christ as his Lord [Rom. 13:14] and is renewed in the spirit of his mind [Eph. 4:23] Those who have been baptized into Christ, have put on Christ [Gal. 3:27] They are born again, incorruptible, by the Word of God, which lives forever [1 Pet. 1:23] and they take glory in nothing except the Cross of the Lord Jesus Christ. The world is dead to them, and they are dead to the world [Gal. 6:14] for in Christ, it doesn't mean anything to be circumcised or uncircumcised, but rather the man must be a new creature. [Gal. 5:6]

	\section{Concerning the Unity of the Saints with Christ}

	He who santifies, and those who are santified, are in unity [Heb. 2:11] For by the great promises that were made to them, they have become divine in nature [2 Pet. 1:4]. Christ prayed for this, that they could all be in unity, as the Father is in Christ, and Christ is in the Father. Christ prayed that He and the Father could be one within them; and Christ gave them the glory he got from the Father, so that they could be unified just as the Father and Christ are unified: Christ is in the Saints, and the Father is in Christ, so that they become a perfect union. [John 17:21-23]

	\section{Concerning the Universal Love and Grace of God to all}

	God so loved the world that he gave his only Son, that whosoever believes in him would not die, but have eternal life [John 3:16] and God showed us His love by sending his only Son, so that we could live through Him [1 John 4:9] so that if any man sins, Christ is his advocate with the Father; and Christ is the atonement for our sins, and not just ours, but the sins of the whole world [1 John 2:1-2] For by God's grace, Christ has tasted death for every man [Heb. 2:9] and gave himself as a ransom for everyone, to be testified in due time [1 Tim. 2:6], wanting every man to be saved and to come to know Christ. [1 Tim. 2:4] Christ did not want anyone to perish, but rather wanted all men to repent. [2 Pet. 3:9] God did not send his Son to condemn the world, but so that through Him, the world could be saved. [John 3:17] Christ came as a Light into the world, that whoever believes in him would not abide in darkness. [John 12:46] Therefore, just as judgement came upon humanity due to one man's sin, eternal life came to humanity through one man's righteousness. [Rom. 5:18]

	\section{Concerning the Light that enlightens every man}

	The gospel was preached to every creature in the world [Col. 1:23] and that gospel is God's power to save those who believe [Rom. 1:16] and if this gospel is unseen, it is unseen by the lost, whose unbelieving minds have been blinded by God, or else the Light of the glorious gospel of Christ would shine into them [2 Cor. 4:3-4] This is the condemnation: Light came into the world, but men love darkness rather than Light, because their deeds are evil [John 3:19]. This Light was the true Light that enlightens every man that comes into the world [John 1:9]. By this light, all things that can be reproved are made manifest: for whatever is manifested is Light [Eph. 5:13] Everyone who does evil hates the Light and does not come into the Light, or else his deeds would be reproved; but he who lives truthfully comes to the Light so his deeds can be shone to be brought about by God. [John 3:20-21] Those who walk in the Light just as Christ is in the Light have fellowship with one another, and Christ's blood cleanses them from all sin [1 John 1:7] Therefore, we should believe in the Light that we have, so we can become children of the Light [John 12:36] Therefore, today, if we hear God's voice, we must not harden our hearts [Heb. 4:7] for Christ wept over Jerusalem, saying, "If only you had known, even now,  the things of your peace. But now, they are hidden from you." [Luke 19:42] And he would have gathered Jerusalem's children like a hen gathering her chicks, but they would not [Matt. 23:37] for the stubborn always resist the Holy Spirit [Acts 7:51] and rebel against the Light. [Job 24:13] Therefore, God's Spirit will not aways be with man [Gen. 6:3] for God's wrath is shown against all ungodliness and unrighteousness of men, who believe the truth itself to be unrighteousness. [Rom. 1:18] His wrath comes upon them because God has shown them the things of God [Rom. 1:19] and the Spirit is given to every man for his benefit [1 Cor. 12:7] Because God's saving grace has appeared to everyone, teaching us to deny ungodliness and worldly lusts, and instead live soberly, righteously, and Godly in this world [Tit. 2:11-12] This Word of His grace builds up and gives an inheritance to those who are santified [Acts 20:32] For God's Word is quick and powerful, sharper than any sword, able to divide the soul and spirit from the body, and able to discern the thoughts and intents of the heart [Heb. 4:12] This is that Word of Phrophecy, that we would do well to heed, like a Light shining in darkness, until the day dawns, and the day-star arises in the heart [2 Pet. 1:19] This is the Word of faith which the apostles preached, which is in the mouth and the heart [Rom. 10:8] God, who commanded Light to shine out of darkness, has shined in our hearts, to give the Light of the knowledge of God's glory in the face of Jesus Christ [2 Cor. 4:6] But we have this treasure in our earthly bodies, so that the excellency of this power comes from God [2 Cor. 4:7] and not from us, for the Kingdom of God can not be seen, but is within us. [Luke 17:20-21]

	\section{Concerning Faith and Justification}

	Faith is the substance of things hoped for, and the evidence of what is not seen [Heb. 11:1] Without this faith, it is impossible to please God [Heb. 11:6] Therefore, we are justified by faith, which works through love. [Gal. 5:6] Faith without works is dead, but by works, faith is made perfect [Jas. 2:22, 26] No one is justified by the law [Rom. 3:20] nor by righteous works. Instead, through God's mercy, we are saved by being regenerated and renewed in the Holy Spirit. [Tit. 3:5] For we are washed, santified, and justfied by both the name of Lord Jesus, and by the Spirit of our God. [1 Cor. 6:11]

	\section{Concerning Good Works}

	If we live according to our flesh, we will die, but if through the Spirit, we do away with the deeds of the body, we will live [Rom. 8:13] Those who believe in God must do good works [Tit. 3:8] for God will give every man according to his deeds according to his righteous judgment, who seek eternal life [Rom. 2:6-7] They are considered worth of the Kingdom of God [2 Thess. 1:5] and do not lose faith, which offers a great reward. [Heb. 10:35] Blessed are those who keep His commandments, for they have a right to the Tree of Life, and can enter through the gates to the city. [Rev. 22:14]

	\section{Concerning Perfection}

	Sin has no power over those who are under God's grace [Rom. 6:14] Those who are in Christ are not condemned, for they don't serve the flesh, but the Spirit, which makes them free from sin and death. [Rom. 8:1-2] Those who are dead to sin and alive to righteousness, are made free from sin and become servants of righteousness [Rom. 6:2,18] Therefore, we ought to be perfect as our heavenly Father is perfect [Matt. 5:48] Christ's burden is easy and light [Matt. 11:30] and His commandments are not difficult [1 John 5:3] and whoever will enter into life must keep the commandments. [Matt 19:17] We know God if we keep His commandments. [1 John 2:3]. If someone says they know God but does not keep His commandments is a liar, and the truth is not in him [1 John 2:4]. Whoever abides in Christ does not sin; whoever sins has not known Christ [1 John 3:6]. Don't be fooled: he who does righteousness is righteous, and he who sins is of the Devil. Whoever is born of God does not commit sin, for the seed of God remains in him, and he cannot sin, because he is born of God [1 John 3:7-9] Not everyone who says, "Lord! Lord!" will enter the Kingdom of Heaven, except for he who does the will of God in Heaven. [Matt 7:21] Being circumised or uncircumsized isn't important, only keeping God's commandments. [1 Cor. 7:19]

	\section{Concerning Perseverance and Falling from Grace}

	We need to make sure that we are truly called by Christ, because if we do, we will never fall away. [2 Pet. 1:10] Even Paul made sure to examine himself, so that when he preached to others, he himself did not become cast waya [1 Cor. 9:27] So let's be careful, making sure there is not evil unbelief in our hearts, which would separate us from the living God [Heb. 3:12] Likewise, let us work to keep from setting an example of unbelief that would cause others to fall. [Heb. 4:11] It is impossible for those who once had God's Light, and tasted the heavenly gift, and partook of the Holy Ghost and the Word of God and the powers of the world to come, if they sould fall away, to come again unto repentance [Heb. 6:4-6] because a man who doesn't abide in Christ is cast away and withers [John 15:16] Yet for those who overcome, He will make them pillars in the temple of God, and they will never leave Him [Rev. 3:12] and nothing will be able to separate them from the love of God, which is in Jesus Christ. [Rom. 8:38]

	\section{Concerning the Church and Ministry}

	God's church is the foundation of truth [1 Tim. 3:15], of which Christ is the head [Col. 1:18], and from which the church, nourished by ministry, grows. [Col. 2:19] God's church consists of those who are sanctified in Jesus Christ [1 Cor. 1:2] who, when he ascended up to heaven, gave men gifts: he made some apostles, some prophets, some evangelists, and some pastors and teachers, in order to perfect the saints for the work of ministry [Eph. 4:8, 11-12]. They ought to be blameless, vigilant, sober, well-behaved, hospitable, and a good teacher; they must not be drunken, or violent, or greedy, but patient.  They must not fight or covet. [1 Tim. 3:2-3] They must love good men, be sober, just, holy, temperate, and faithful, as they have been taught, so that through sound doctrine, they can exhort and convince others. [Tit. 1:8-9] They must be careful that they and their flock, which the Holy Spirit has entrusted them to oversee, feed God's church. [Acts 20:28] They must take this oversight willingly, not for worldly gain but of a sound and ready mind, not being a lord over God's heritage, but by being examples to the flock. [1 Pet. 5:2-3] Those elders who rule well are worthy of double the honor, especially if they labor in the Word and doctrine [1 Tim. 5:17], and they should bee esteemed highly in love for the sake of their works [1 Thess. 5:13]. Just as every man has received the Gift, the gift must be ministered; if any man speaks, let him speak for God; if any man ministers, let him do so according to the ability God has given him. [1 Pet. 4:10-11] He must preach the gospel, not with the wisdom of Words (or else the Cross of Christ has no effect) [1 Cor. 1:17] or with the words of men's wisdom, but by demonstrating the Spirit and power of God, so that the faith doesn't rely on the wisdom of men, but on God's power. [1 Cor. 2:4-5] Those who are perfect speak wisdom, but it is not the wisdom of this world or of the rulers of this world, which comes to nothing; rather, they speak the mysterious wisdom of God, even the hidden wisdom, which God ordained before the world to their glory [1 Cor. 2:6-7] They don't speak in words taught by men, but in the wisdom of the Holy Spirit [1 Cor. 2:13] It isn't their own words that they speak, but the words of the Holy Spirit, who speaks through them. [Matt. 10:20] If they speak spiritual things, they ought to live by its example, because the Lord commands that those who preach the gospel live according to the gospel, for Scripture says, "You must not muzzle the mouth of the Ox that spreads the corn, and the laborer deserves his reward [1 Cor. 9:11, 14, 9]. However, a requirement is put on them that they must preach the gospel, and their reward is that when they preach the gospel, the gospel is beyond criticism [1 Cor. 9:16-18] They must not covet any man's silver, or gold, or clothing, but they must work for their own necessities, so they can support the weak, remembring the words of Jesus, who said it is more blessed to give, than to receive [Acts 20:33-35] They are not lik greedy dogs that never have enough [Isa. 56:11] or like shepherds that only look out for themselves, [Isa. 56:11] feeding themselves and not the flock [Ezek. 34:8] causing people to stumble, crying for peace but fighting all who don't give to them, teaching for hire, and prophesying for money [Mic. 3:5,11] They are not like those who teach what is wrong in order to make money. [Tit. 1:11], which run greedily after the error of Balaam for their reward, loving to profil from their evil [2 Pet. 2:15] Through jealousy and greed, they lie, profiting off of damned souls [2 Pet. 2:3], corrupt in their minds, lacking the truth, and believing that material gain is godliness. [1 Tim. 6:5] Rather, they know that being content in godliness is its own reward [1 Tim. 6:6] and are content merely to have food and clothing [1 Tim. 6:8]

	\section{Concerning Worship}

	The hour will come and has come when true worshippers will worship the Father in Spirit and in Truth, for the Father wants them to worship him. [John 4:23] God is a Spirit, and those who worship him must worship him in Spirit and in truth [John 4:24] because to Lord is near to all who call upon Him in Truth [Psal. 145:18] He is not wicked, but rather He hears the prayers of the righteous [Prov. 15:29] And this is the confidence that we have in Him: If we ask anything according to His will, he hears us [1 John 5:14] What is His will?  We must pray with the Spirit, and with understanding; we must sing with the Spirit, and with understanding. [1 Cor. 14:15] Likewise, the Spirit helps us with our weaknesses, for we don't know what we should pray for (though we should); but the Spirit speaks to us with unspoken callings. He who searches his heart knows the mind of the Spirit, because the Spirit intercedes for the saints according to God's will. [Rom. 8:26-27]

	\section{Concerning Baptism}

	Just as there is one Lord and one faith, so is there one baptism [Eph. 4:5] which saves us. This is not the putting away of the filth of our flesh, but rather it is the answer of our conscience toward God, by the resurrection of Jesus Christ [1 Pet. 3:21-22] For John baptized with water, but Christ baptized with the Holy Spirit and with fire. [Matt. 3:11] Therefore everyone who is baptized into Jesus Christ is also baptized into his death, and they are buried with him by baptism into death, that just as Christ was raised up from the dead by the glory of the Father, they too should walk as if resurrected [Rom. 6:3-4], since they have taken on Christ. [Gal. 3:27]

	\section{Concerning Eating of Bread and Wine; Washing of omne another's Feet; Abstaining from things Strangled, and from Blood; and Anointing of the Sick with Oil}

	The Lord Jesus, on the same night that He was betrayed, took bread, and when He had given thanks, he broke it and said, "Eat this, my body, which is broken for you; do this in remembrance of me." In the same manner, he took the cup, saying, "This cup is the New Testament in my blood; drink it in remembrance of me. When you eat this bread and drink this cup, you show the Lord's death until he comes again. [1 Cor. 11:23-26] Jesus, knowing that the Father had placed everything in His hands, and that he came from God, and went to God, he arose from supper, laid aside his garments, girded himself with a towel, and began washing his disciples' feet, then wiped them with the towel. After he had washed their feet, and had taken his garments and sat down again, he said to them, "Do you know what I have done for you?  You call me Master and Lord, and you are right, because I am. Well, if I, your Lord and Master, have washed your feet, you should also wash each other's feet, for I have set an example, and you should do as I have done for you. [John 13:3-5, 12-15] It seemed right to the Holy Spirit and to us, not to burden you with anymore than these necessary things, that you abstain from meats offered to idols, from blood, and from things strangled, and from fornication. If you keep from doing these things, you have done well. [Acts 15:28-29] If any man among you is sick, he should call for the church elders, and they should pray over him, anointing him with oil. [James 5:14]

	\section{Concerning the Liberty of such Christians as are come to know the Substance, as to the Using, or not Using of these Rites, and of the Observation of Days}

	The Kingdom of God is not meat or drink, but righteousness, peace, and joy in the Holy Spirit. [Rom. 14:17] Therefore, no man should judge us in meat or drink, or in our observation of a holy day, of of the new moon, or of the Sabbath. [Col. 2:16] For if we are dead with Christ from the ways of the world, why are we subject to these ordinances as if we live in the world? We should not touch, or taste, or handle according to the commandments and doctrines of men. [Col. 2:20-22] Now, after we know God, or rather are known by Him, why should we turn to these weak and useless rituals, or desire to be bound in observation of days, and months, and times, and years, as if God's labors were in vain? [Gal. 4:9-11] If one man celebrates a day above others, another celebrates all days the same. Every man should be assured in his own mind: If he recognizes a day, he should celebrate it in honor of God; if he does not celebrate it in honor of God, he does not recognize it at all. [Rom. 14:5,6]

	\section{Concerning Swearing, Fighting and Persecution}

	It has been said by those of old, "You must not swear by yourself, but swear your oaths to the Lord," but Christ says to us, "Do not swear at all, by Heaven, for it is God's throne, nor by Earth, for it is his footstool. Do not swear by Jerusalem, for it is the city of the great king, nor by your head, because you can't make a single hair white or black. Rather, let your yes be yes, yes; or no, no; for whatever you say beyond this comes from evil. [Matt. 5:33-37] And James tells us, "Above all, do not swear, not by Heaven, or by Earth, or by any other oath; but let your yes by es, and you no be no, or else you will be condemned. [Jas. 5:12] Though we walk in the flesh, we must not go to war for the flesh, for our weapons are not carnal, but are mighty through God, capable of destroying strongholds, casting down imaginations and everything that exalts itself against the knowledge of God, and bringing every thought to obedience of Christ. [2 Cor. 10:3-5] Wars and fighting comes from lusts that battle within men. [Jas. 4:1-2] Christ instead commands, "Do not resist evil, but whoever strikes your right cheek, turn the other also. [Matt. 5:39] Christians are lambs among wolves [Luke 10:3] and therefore they are hated by all men for Christ's sake. [Matt. 10:22] All that live Godly lives in Jesus Christ must suffer persecution [2 Tim. 3:12] They are blessed, for theirs is the Kingdom of Heaven. [Matt. 5:10] For though they have lost their lives, they have also saved them. [Matt. 16:25] Because they have confessed Christ before men, he will also confess them before the angels of God. [Luke 12:8-9] Therefore, we should not be afraid of those who kill the body but cannot kill the soul. Rather, we should fear He who can destroy both the soul and the body in hell. [Matt. 10:28]

	\section{Concerning Magistracy}

	Let every soul be subject to the higher powers, for there is no power but of God. The powers that be are thus ordained by God. Therefore, whoever resists the power, resists the ordinance of God, and those who resist shall by damned, for rulers do not oppose good works, but rather oppose evil. Thus, do not be afraid of such power; do what is right, and you shall have the praise of such powers, for he is a minister of God for your benefit. But if you do evil, be afraid, for he does not bear the sword in vain, because he is the minister of God, who takes revenge upon those that do evil. Therefore we must be obedient, not only in fear of their wrath, but for the sake of our own conscience. This is also why we pay taxes, for they are God's ministers, continually subject to this: Render to all what is due to them, give tribute to whom tribute is due, pay custom to whom custom is due, fear to whom fear is due, and honor to whom honor is due. [Rom. 13:1-7] Therefore, we must submit ourselves to every ordinance of man for the Lord's sake, whether to the king, or to governors, as they are sent by God to punish evil, and to praise those who do good, for this is God's will, that by doing what is right, we will silence the ignorance of fools. [1 Pet. 2:13-15] That being said, God expects us to obey Him before them, [Acts 4:19] and though they command us not to teach in Christ's name, we ought to obey God rather than men. [Acts 5:28-29]

	\section{Concerning the Resurrection}

	There will be a resurrection of the dead, both just and unjust. [Acts 24:15] Those who have done good will be resurrected unto life. Those who have done evil will be resurrected unto damnation. [John 5:29] Flesh and blood cannot inherit the Kingdom of God, just as corrpution cannot inherit incorruption. [1 Cor. 15:50] The body that is buried is not the body that is resurrected. Rather, God will give us the body that pleases Him. The body is buried in corruption but raised in incorruption. It is buried in dishonor but raised in glory. It is buried in weakness but raised in power. It is buried a natural body but raised a spiritual body. [1 Cor. 15:37-38, 42-44]



	\theendnotes
	\setcounter{endnote}{0}



	\end{document}