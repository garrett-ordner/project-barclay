%Section Template
\documentclass[../main.tex]{subfiles}

\begin{document}
	

	\section*{}
	Q. Did Christ promise his disciples that he would come again?

	A. I will not leave you comfortless; I will come unto you. [John 14:18]\qVerse{John}{14}{18}

	Q Was this just a special promise to his disciples? Or does it extend to all the saints?

	A. God says that while he dwells in a high and holy place, with him is a contrite and humble spirit. [Isa. 57:15]\qVerse{Isaiah}{57}{15}

	You are the temple of the living God, and God says He will dwell in you and walk among you. [2 Cor. 6:16]\qVerse{IICorinthians}{6}{16}

	Christ knocks at the door, and if any man hears him and opens the door, Christ will come in. [Rev. 3:20]\qVerse{RevelationofJohn}{3}{20}

	Q. Does the apostle Paul speak of the Son of God's being revealed in him?

	A. God revealed his Son in me so I could preach among the heathen. [Gal. 1:15-16]\qRange{Galatians}{1}{15}{16}

	Q. Does this mean that we need to know Christ within?

	A. Examine yourselves and whether you are in the faith. Don't you know that Christ is in you, unless you are reprobates? [2 Cor. 13:5]\qVerse{IICorinthians}{13}{5}

	Q. Was Paul telling the truth when he said that the inward birth of Christ should be brought forth in believers?

	A. My children, I am in pain until Christ is formed in you. [Gal. 4:19]\qVerse{Galatians}{4}{19}

	Q. What does Paul say about the need of this inward knowledge of Christ, and of man becoming a new creature?

	A. Because from now on we know no one in a worldy sense. Though we have known Christ in such a sense, we no longer do. If any man is in Christ, he is a new creature. [2 Cor. 5:16-17]\qRange{IICorinthians}{5}{16}{17}

	When you learned of Christ, you learned to put away your old corrupt self, and to become a new, Godly person. [Eph. 4:20-24]\qRange{Ephesians}{4}{20}{24}

	Q. So this inward Christ is the hope that Paul was preaching about?

	A. God would make known the riches of the glory of this mystery, Christ within you, to the Gentiles. [Col. 1:27]\qVerse{Colossians}{1}{27}

	Q. Does Paul encourage undergoing this new birth anywhere else?

	A. Put on the Lord Jesus Christ, and don't allow your flesh to fulfill its lusts. [Rom. 13:14]\qVerse{Romans}{13}{14}

	Q. Does Paul write to any of the saints about getting rid of the old man and putting on the new?

	A. Those of you who have been baptized to into Christ have put on Christ. [Gal. 3:27]\qVerse{Galatians}{3}{27}

	You've put off the old man with his worldly deeds, and have instead put on the new man, renewed in the knowledge of God. [Col. 3:9-10]\qRange{Colossians}{3}{9}{10}

	Q. What does Christ say about thie necessity of this new birth?

	A. Jesus said to him, "Unless a man is born again, he cannot see the kingdom of God." [John 3:3]\qVerse{John}{3}{3}

	Q. What seed does this birth come from?

	A. We are born again of the everlasting Word of God, not some corruptible see. [1 Pet. 1:23]\qVerse{IPeter}{1}{23}

	Q. What does Paul say about this rebirth?

	A. I am crucified with Christ, and I no longer live, but Christ lives in me. [Gal. 2:20]\qVerse{Galatians}{2}{20}

	Q. What does the cross of Christ preach?

	A. To the damned, it is foolishness, but to us who are saved, it is the power of God. [1 Cor. 1:18]\qVerse{ICorinthians}{1}{18}

	Q. How did the Cross affect the apostle Paul? And how much did he prefer this rebirth over any outward and visible ordinances and traditions?

	A. God forbids me from glory, except the glory of the cross of Christ, for whom I disown the world, and because of whom the world disowns me. In Jesus Christ, circumcision and uncircumcision don't matter, only a new man. [Gal. 6:14-15]\qRange{Galatians}{6}{14}{15}

	Q. What does Christ say about His unity with the saints?

	A. On that day, you will know that I am in my Father, and you are in me, and I am in you. [John 14:20]\qVerse{John}{14}{20}

	Abide in me, and I in you. A branch can't bear fruit by itself; it needs to be with the vine.  You can only bear fruit if you abide in me. If a man abides in me, and I abide in him, he will bear much fruit, because without Me, you can do nothing. [John 15:4-5]\qRange{John}{15}{4}{5}

	I don't just pray for these alone, but also for those who will believe in me through their words. They will all be in unity, and you, Father, are with me, and I with you, so will they be one with us, so the world will believe that you sent me. I have given to them the glory you gave Me, so that they can be one with us. I am in them, and you are in me, so they can be made perfect in unity, and the world will know that you sent me, and that you love them as you love me. [John 17:20-23]\qRange{John}{17}{20}{23}

	Q. What does Paul say about this unity?

	A. He who sanctifies and those who are sanctified are all in unity, so He is not ashamed to call them brothers. [Heb. 2:11]\qVerse{Hebrews}{2}{11}

	Q. What does the apostle Peter say about this?

	A. We are given great and precious promises that we can partake in God's divine nature, escaping the corruption of the world. [2 Pet. 1:4]\qVerse{IIPeter}{1}{4}


\end{document}
