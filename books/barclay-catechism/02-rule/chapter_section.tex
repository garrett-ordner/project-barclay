%Section Template
\documentclass[../main.tex]{subfiles}

\begin{document}
	

	\section*{}
	Q. Since Christ reveals the knowledge of God through the Spirit, are we led to understand the gospel through the Spirit as well?
	
	A. If the Spirit dwells in you, then you are not of the flesh, but the Spirit. If a man lacks the Spirit of Christ, he does not belong to Christ. [Rom. 8:9]\qVerse{Romans}{8}{9}
	
	Those who are led by the Spirit of God are the sons of God. [Rom. 8:14]\qVerse{Romans}{8}{14}
	
	Q. In that case, is it an inward principle that guides and instructs Christians?
	
	A. If you have been anointed by Christ, then you don't need the teachings of men, as that anointing teaches you the truth, and thus you abide in Christ. [1 John 2:27]\qVerse{IJohn}{2}{27}
	
	As for brotherly love, no one needs to exhort you to love your brother, as God teaches you to love one another. [1 Thess. 4:9]\qVerse{IThessalonians}{4}{9}
	
	Q. From that answer, I understand that the inward anointing and rule is what teaches Christians; is this the basis of the New Covenant?
	
	A. This is the Covenant that God has made with Israel: God will put his laws into their mind and write them on their hearts. He will be their God, and they will be His people. They won't need to teach their neighbors and brothers, for everyone will know God, from the least to the greatest. [Heb. 8:10-11] \qRange{Hebrews}{8}{10}{11}
	
	They will all learn of God. [John 6:45]\qVerse{John}{6}{45}
	
	Q. Did Christ promise that the Spirt would both be with His disciples, and be within them?
	
	A. Christ said his Father would give us another comforter, the Spirit of truth, that would be with us forever, but that worldly men could not receive, because they don't see or know it. We Christians, however, do know the Spirit, for it dwells with us, and is within us. [John 14:16,17]\qRange{John}{14}{16}{17}
	
	Q. Why were the Scriptures written?
	
	A. The Scriptures were written to teach us that our patience and the comfort of the Scriptures would give us hope. [Romans 15:4]\qVerse{Romans}{15}{4}
	
	Q. How are the Scriptures helpful?
	
	A. The Holy Scriptures teach us of salvation through faith in Christ. Scripture is inspired by God, and is useful to teach us doctrine, to teach us what we should disapprove, the correct those in error, and to instruct us in righteousness, so that we may be perfect in the things we do. [2 Tim. 3:15-17]\qRange{IITimothy}{3}{15}{17}
	
	Q. What makes the Scriptures so perfect?
	
	A. The Scriptures were not written by men according to their own will, but rather these men were writing what the Holy Spirit told them to. [2 Pet. 1:20-21]\qRange{IIPeter}{1}{20}{21}
	
	Q. So the Scriptures should be respected because they came from the Spirit, and because these Scriptures testify that the Spirit leads us to Truth, not the Scriptures themselves. How does Christ command us to read the Scriptures?
	
	A. Christ tells us to read the Scriptures, because while we may think we have eternal life in them, they actually testify of Christ. [John 5:39]\qVerse{John}{5}{39}
	
	Q. I understand that a past generation greatly respected the Scriptures, but would not believe in Christ or be guided by the Holy Spirit that the Scriptures testified to. How does Christ explain this?
	
	A. Christ says he doesn't have to accuse us of unbelief, as Moses himself does that. If we believed Moses, we would believe Christ, because Moses wrote of Christ. If we won't believe the writings of Moses, how can Christ expect us to believe His own words? [John 5:45-47]\qRange{John}{5}{45}{47}
	
	Q. So then, how do we account for their unbelief, in spite of their claim to follow the Scriptures?
	
	A. They don't understand the Scriptures, and so their claim to follow the Scriptures is their own undoing. [2 Pet. 3:16]\qVerse{IIPeter}{3}{16}
	

\end{document}